% \section{Related works} %放正文
\section{Related works} %放附录
There are many works that have attempted to address the non-iid problem in federated learning.
\fedavg, first presented by \citet{mcmahan2017communication}, has been demonstrated to have issues with convergence when working with non-iid data. \citet{zhao2018federated} depict the non-iid trap as weight divergence, and it can be reduced by sharing a small set of data. However, in traditional federal setting, data sharing violates the principle of data privacy. \citet{karimireddy2021scaffold} highlight the phenomenon of ``client drift" that occurs when data is heterogeneous (non-iid), and uses control variates to address this problem. However, using Scaffold in cross-device FL may not be effective, as it requires clients to maintain the control variates, which may become outdated and negatively impact performance. \citet{li2020federated} propose FedProx that utilizes a proximal term to deal with heterogeneity. \looseness=-1  
% \citet{karimireddy2020mime} propose

In addition to these works, some research has noticed the presence of period drift, but have not specifically addressed it in their analysis. For example, \citet{cho2022towards, fraboni2023general} investigate the problem of biased client sampling and proposes an sampling strategy that selects clients with large loss. However, active client sampling can potentially alter the overall data distribution by having unrandom clients participation, which can raise concerns about fairness. Similarly, \citet{yao2019federated} propose a meta-learning based method for unbiased aggregation, but it requires training the global model on a proxy dataset, which may not be feasible in certain scenarios where such a dataset is not available. \citet{zhu2022diurnal} observe that the data on clients have periodically shifting distributions that changed with the time of day, and model it using a mixture of distributions that gradually shifted between daytime and nighttime modes. \citet{guo2021towards} study the impact of time-evolving heterogeneous data in real-world scenarios, and solve it in a framework of continual learning. Although these two papers define similar terms, they focus on the case of client data changing over time. However, in this paper, we find that even if the distribution of client data remains unchanged, period drift can seriously affect the convergence of FL.