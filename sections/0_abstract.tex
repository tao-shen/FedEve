\begin{abstract}
    Federated learning (FL) is a machine learning paradigm that allows multiple clients to collaboratively train a shared model without exposing their private data. Data heterogeneity is a fundamental challenge in FL, which can result in poor convergence and performance degradation. \textit{Client drift} has been recognized as one of the factors contributing to this issue resulting from the multiple local updates in FedAvg. However, in cross-device FL, a different form of drift arises due to the partial client participation, but it has not been studied well. This drift, we referred as \textit{period drift}, occurs as participating clients at each communication round may exhibit distinct data distribution that deviates from that of all clients. It could be more harmful than client drift since the optimization objective shifts with every round.
    In this paper, we investigate the interaction between period drift and client drift, finding that period drift can have a particularly detrimental effect on cross-device FL as the degree of data heterogeneity increases. To tackle these issues, we propose a predict-observe framework and present an instantiated method, FedEve, where these two types of drift can compensate each other to mitigate their overall impact. We provide theoretical evidence that our approach can reduce the variance of model updates. Extensive experiments demonstrate that our method outperforms alternatives on non-iid data in cross-device settings.
\end{abstract}