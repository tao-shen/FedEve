\usepackage{wrapfig}
% my configurations
\usepackage{amsmath}
\usepackage{graphicx}
\usepackage{amsmath}
\usepackage{amssymb}
\usepackage{mathtools}
\usepackage{amsthm}
\usepackage{subfigure}
\usepackage{wrapfig}
\usepackage{algorithm}
\usepackage{algorithmic}
\usepackage{enumitem}
\usepackage{booktabs}
\usepackage{colortbl}
\usepackage{array}
\usepackage{multirow}
\usepackage{tikz}
\usepackage{xcolor}
\usetikzlibrary{tikzmark}

\usetikzlibrary{shapes,arrows}
\definecolor{Gray}{gray}{0.9}
\definecolor{PeriodDriftColor}{HTML}{CF96B0}
\definecolor{ClientDriftColor}{HTML}{929FDA}

\makeatletter
\def\input@path{{./sections/}{../sections/}{./figures/}{../figures/}{./}{../}}
\makeatother
\graphicspath{{./figures/}{../figures/}}
 
\newcommand{\fedavg}{\textsc{FedAvg}~}
\newcommand{\fedeve}{\textsc{FedEve}~}
\newcommand{\fedavgm}{\textsc{FedAvgM}~}
\newcommand{\fedprox}{\textsc{FedProx}~}
\newcommand{\scaffold}{\textsc{Scaffold}~}
\newcommand{\fedopt}{\textsc{FedOpt}~}
\newcommand{\feddf}{\textsc{FedDF}~}
\newcommand{\fedmeta}{\textsc{FedMeta}~}
\newcommand{\fedleo}{\textsc{FedLeo}~}
\newcommand{\red}[1]{\textcolor{red}{#1}}
 
%%%%%%%%%%%%%%%%%%%%%%%%%%%%%%%%
% THEOREMS
%%%%%%%%%%%%%%%%%%%%%%%%%%%%%%%%
\theoremstyle{plain}
\newtheorem{theorem}{Theorem}[section]
% \newtheorem{proposition}[theorem]{Proposition}
\newtheorem{lemma}[theorem]{Lemma}
% \newtheorem{corollary}[theorem]{Corollary}
\theoremstyle{definition}
\newtheorem{definition}[theorem]{Definition}
\newtheorem{assumption}[theorem]{Assumption}
% \theoremstyle{remark}
% \newtheorem{remark}[theorem]{Remark}

 
\setlength{\parskip}{2.5pt plus3pt minus3pt}
% 1.2pt:基础段落间距为 1.2 点(约 0.42 毫米)。
% plus3pt:允许 LaTeX 在排版需要时增加最多 3pt 的弹性间距(例如避免页面底部留白过多)。
% minus2.5pt:允许 LaTeX 在排版需要时减少最多 2.5pt 的弹性间距(例如避免内容被截断)。
% \setlength{\textfloatsep}{3pt plus2pt minus2pt} % 顶部/底部浮动体
% \setlength{\intextsep}{3pt plus2pt minus2pt}      % 中间浮动体
% 减少图表与上下文的间距(例如设为 10pt + 弹性值)

% 调整图片标题与内容的间距
% \setlength{\abovecaptionskip}{-10pt}  % 标题上方间距
% \setlength{\belowcaptionskip}{-10pt}   % 标题下方间距

% % 调整表格标题与内容的间距(表格标题通常在表格上方)
% \setlength{\abovecaptionskip}{-10pt}   % 表格标题上方间距
% \setlength{\belowcaptionskip}{-10pt}   % 表格标题下方间距